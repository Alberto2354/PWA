\documentclass{article}
\usepackage{graphicx}
\begin{document}
	\begin{titlepage}
		\centering
		{\bfseries\LARGE Universidad Tecn\'ologica de Tijuana \par}
		\vspace{1cm}
		\vspace{3cm}
		{\scshape\Huge Title \par}
		\vspace{1cm}
		{\itshape\Large PWA concepts and features \par}
		\vfill
		{\Large Author: \par}
		{\Large Reyes Navarro Juan Alberto \par}
		\vfill
		{\Large Group: \par}
		{\Large 10A \par}
		\vfill
		{\Large Wednesday, January 10, 2024 \par}
	\end{titlepage}
	\newpage
	\section{Concept and characteristics of PWA}
		\subsection{Concept}
	Are web applications that use web platform technologies to provide a user experience similar to that of a native app. PWAs can run on multiple platforms and devices from a single codebase, just like websites. They can be installed on the device, can operate while offline and in the background, and can integrate with the device and with other installed apps.
		\subsection{characteristics}
		\begin{flushleft}
			\textbf{Responsive:} PWAs must adapt automatically to any format, browser, or device.
		
		\textbf{Updated}: PWAs always show their latest version to the user, with the use of automatic updates.
		
		\textbf{Secure:} PWAs use the secure HTTPS protocol to ensure safe access and prevent content manipulation.
		
		\textbf{Fast:} PWAs have optimized loading and navigation speeds.
		\end{flushleft}
		\section{Difference between A web application, a service-oriented application and a progressive application}
		\subsection{Web application}
		
	\begin{flushleft}
		\textbf{Characteristics:} A web application is an application that runs in a web browser.
		
		\textbf{Types:}Document-centric web application, Interactive web application, Transactional web application, Workflow-based web application, Collaborative web application, Portal-oriented web application, Ubiquitous web application, Knowledge-based web application.
		
		\textbf{Advantages:}	Web applications are easy to develop and maintain. They are also platform-independent and can be accessed from any device with a web browser.
		
		\textbf{Disadvantages:}	Web applications are less responsive than native applications. They are also less secure and can be vulnerable to attacks.
	\end{flushleft}
	
		\subsection{Service-Oriented Application}
	
	\begin{flushleft}
		\textbf{Characteristics:} A service-oriented application is an application that is composed of services that communicate with each other.
		
		\textbf{Types:}Business Process Management (BPM), Enterprise Service Bus (ESB), Service Registry, Service Repository, Service Broker, Service Monitor, Service Provisioning, Service Level Agreement (SLA).
		
		\textbf{Advantages:}Service-oriented applications are scalable and can be easily integrated with other applications. They are also reusable and can be easily maintained.
		
		\textbf{Disadvantages:}	Service-oriented applications can be complex and difficult to develop. They can also be difficult to test and debug.
	\end{flushleft}
	
		\subsection{Progressive Application}
	
	\begin{flushleft}
		\textbf{Characteristics:}A progressive application is a web application that behaves like a native application.
		
		\textbf{Types:}Single Page Application (SPA), Server-Side Rendering (SSR), Static Site Generation (SSG), Hybrid Application.
		
		\textbf{Advantages:}Progressive applications provide a native-like experience to users. They are also responsive and can work offline.
		
		\textbf{Disadvantages:}	Progressive applications can be difficult to develop and maintain. They can also be less responsive than native applications.
	\end{flushleft}
	\section{Tools for developing and executing PWAs}
			\subsection{Visual Studio Code}
			A lightweight, cross-platform source code editor that provides a rich development experience for building web and cloud applications.
			
\textbf{Requirements for Visual Studio Code}

1.6 GHz or faster processor
1 GB of RAM

\textbf{Platforms}


Windows 10 and 11 (64-bit)

macOS versions with Apple security update support.

Linux (Debian): Ubuntu Desktop 18.04, Debian 10

Linux (Red Hat): Red Hat Enterprise Linux 7, CentOS 7, Fedora 35

\subsection{Node.js:}
A JavaScript runtime built on Chrome’s V8 JavaScript engine that allows you to run JavaScript on the server-side.

\textbf{Requirements for Node.js}
Node.js runs on POSIX-like operating systems, various UNIX derivatives (Solaris, for example) or workalikes (Linux, macOS, and so on), as well as on Microsoft Windows. It can run on machines both large and small, including the tiny ARM devices such as the Raspberry Pi microscale embeddable computer for DIY software/hardware projects.

\subsection{Ionic:}
An open-source SDK based on Apache Cordova and Angular framework

\textbf{Requirements for Ionic}
To get started with Ionic Framework, the only requirement is a Node and npm environment. 

\subsection{React:}
A JavaScript library for building user interfaces 

\textbf{Requirements for React}

Windows XP, Windows 7 (32/64 bit) or higher

Minimum 4 GB RAM and higher.

10 GB available space on the hard disk.

At least one Internet Browser e.g. Chrome, Firefox, Microsoft Edge etc.

Node.js.

Active internet connection minimum speed 512kbps and above.

At least one installed code Editor to test and debug your code e.g. 

\section{Best tools for a pwa development environment} 
\begin{itemize}
	\item Ionic
	\item Polymer
	\item AngularJS
	\item Vue.js
	\item React
	\item Magento PWA Studio
	\item ScandiPWA
	\item Svelte 
	\item Preact 
	\item Lighthouse 
\end{itemize}
\newpage

\begin{figure}
	\centering
	\includegraphics[width=0.8\linewidth]{ref}
\end{figure}


\end{document}
