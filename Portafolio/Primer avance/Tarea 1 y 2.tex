\documentclass{IEEEtran}

\usepackage{graphicx}

\begin{document}
	
	\title{PWA Concepts and Features}
	\author{Reyes Navarro Juan Alberto}
	
	\maketitle
	
	\section{Concept and Characteristics of PWA}
	
	\subsection{Concept}
	
	Progressive Web Applications (PWAs) are web applications that use web platform technologies to provide a user experience similar to that of a native app. PWAs can run on multiple platforms and devices from a single codebase, just like websites. They can be installed on the device, can operate while offline and in the background, and can integrate with the device and with other installed apps.
	
	\subsection{Characteristics}
	
	\begin{itemize}
		\item \textbf{Responsive:} PWAs must adapt automatically to any format, browser, or device.
		\item \textbf{Updated:} PWAs always show their latest version to the user, with the use of automatic updates.
		\item \textbf{Secure:} PWAs use the secure HTTPS protocol to ensure safe access and prevent content manipulation.
		\item \textbf{Fast:} PWAs have optimized loading and navigation speeds.
	\end{itemize}
	
	\section{Difference Between a Web Application, a Service-Oriented Application, and a Progressive Application}
	
	\subsection{Web Application}
	
	\begin{itemize}
		\item \textbf{Characteristics:} A web application is an application that runs in a web browser.
		\item \textbf{Types:} Document-centric web application, Interactive web application, Transactional web application, Workflow-based web application, Collaborative web application, Portal-oriented web application, Ubiquitous web application, Knowledge-based web application.
		\item \textbf{Advantages:} Web applications are easy to develop and maintain. They are also platform-independent and can be accessed from any device with a web browser.
		\item \textbf{Disadvantages:} Web applications are less responsive than native applications. They are also less secure and can be vulnerable to attacks.
	\end{itemize}
	
	\subsection{Service-Oriented Application}
	
	\begin{itemize}
		\item \textbf{Characteristics:} A service-oriented application is an application that is composed of services that communicate with each other.
		\item \textbf{Types:} Business Process Management (BPM), Enterprise Service Bus (ESB), Service Registry, Service Repository, Service Broker, Service Monitor, Service Provisioning, Service Level Agreement (SLA).
		\item \textbf{Advantages:} Service-oriented applications are scalable and can be easily integrated with other applications. They are also reusable and can be easily maintained.
		\item \textbf{Disadvantages:} Service-oriented applications can be complex and difficult to develop. They can also be difficult to test and debug.
	\end{itemize}
	
	\subsection{Progressive Application}
	
	\begin{itemize}
		\item \textbf{Characteristics:} A progressive application is a web application that behaves like a native application.
		\item \textbf{Types:} Single Page Application (SPA), Server-Side Rendering (SSR), Static Site Generation (SSG), Hybrid Application.
		\item \textbf{Advantages:} Progressive applications provide a native-like experience to users. They are also responsive and can work offline.
		\item \textbf{Disadvantages:} Progressive applications can be difficult to develop and maintain. They can also be less responsive than native applications.
	\end{itemize}
	
	\section{Tools for Developing and Executing PWAs}
	
	\subsection{Visual Studio Code}
	
	Visual Studio Code is a lightweight, cross-platform source code editor that provides a rich development experience for building web and cloud applications.
	
	\textbf{Requirements for Visual Studio Code:}
	\begin{itemize}
		\item 1.6 GHz or faster processor
		\item 1 GB of RAM
	\end{itemize}
	
	\textbf{Platforms:}
	\begin{itemize}
		\item Windows 10 and 11 (64-bit)
		\item macOS versions with Apple security update support.
		\item Linux (Debian): Ubuntu Desktop 18.04, Debian 10
		\item Linux (Red Hat): Red Hat Enterprise Linux 7, CentOS 7, Fedora 35
	\end{itemize}
	
	\subsection{Node.js}
	
	Node.js is a JavaScript runtime built on Chrome’s V8 JavaScript engine that allows you to run JavaScript on the server-side.
	
	\textbf{Requirements for Node.js:}
	Node.js runs on POSIX-like operating systems, various UNIX derivatives (Solaris, for example) or workalikes (Linux, macOS, and so on), as well as on Microsoft Windows. It can run on machines both large and small, including the tiny ARM devices such as the Raspberry Pi microscale embeddable computer for DIY software/hardware projects.
	
	\subsection{Ionic}
	
	Ionic is an open-source SDK based on Apache Cordova and Angular framework.
	
	\textbf{Requirements for Ionic:}
	To get started with the Ionic Framework, the only requirement is a Node and npm environment.
	
	\subsection{React}
	
	React is a JavaScript library for building user interfaces.
	
	\textbf{Requirements for React:}
	\begin{itemize}
		\item Windows XP, Windows 7 (32/64 bit) or higher
		\item Minimum 4 GB RAM and higher.
		\item 10 GB available space on the hard disk.
		\item At least one Internet Browser (e.g., Chrome, Firefox, Microsoft Edge, etc.)
		\item Node.js
		\item Active internet connection minimum speed 512kbps and above.
		\item At least one installed code Editor to test and debug your code (e.g., Visual Studio Code)
	\end{itemize}
	
	\section{Best Tools for a PWA Development Environment}
	
	\begin{itemize}
		\item Ionic
		\item Polymer
		\item AngularJS
		\item Vue.js
		\item React
		\item Magento PWA Studio
		\item ScandiPWA
		\item Svelte 
		\item Preact 
		\item Lighthouse 
	\end{itemize}

\begin{thebibliography}{99}
	
	\bibitem{mora2022}
	S. L. Mora.
	\emph{Qué es PWA: características, tecnologías, ventajas y desventajas}.
	\emph{DIGITAL55}, 2022.
	\url{https://digital55.com/blog/que-es-pwa-ventajas-desventajas/}
	
	\bibitem{mdn}
	\emph{Aplicaciones web progresivas}.
	Mozilla Developer Network.
	\url{https://developer.mozilla.org/es/docs/Web/Progressive_web_apps}
	
	\bibitem{klein2021}
	M. Klein.
	\emph{What is a progressive web application?}.
	\emph{Codecademy Blog}, 2021.
	\url{https://www.codecademy.com/resources/blog/what-is-a-progressive-web-application/}
	
	\bibitem{webapp}
	\emph{Web applications}.
	\url{http://cs.uccs.edu/%7Ecs526/jwsdp/docs/tutorial/doc/WebApp.html}
	
	\bibitem{ramirez2018}
	I. Ramírez.
	\emph{¿Qué es una Aplicación Web Progresiva o PWA?}.
	\emph{Xataka}, 2018.
	\url{https://www.xataka.com/basics/que-es-una-aplicacion-web-progresiva-o-pwa}
	
	\bibitem{vidal2022}
	M. Vidal.
	\emph{PWA: Qué son y cómo funcionan las Progressive Web Apps}.
	\emph{Thinking for Innovation}, 2022.
	\url{https://www.iebschool.com/blog/progressive-web-apps-analitica-usabilidad/}
	
	\bibitem{singh2020}
	A. Singh.
	\emph{Categories of web applications | Characteristics of web applications}.
	\emph{Tech Blog}, 2020.
	\url{https://msatechnosoft.in/blog/categories-of-web-applications-characteristics-of-web-applications/}
	
	\bibitem{vscode}
	\emph{Requirements for Visual Studio Code}.
	2021.
	\url{https://code.visualstudio.com/docs/supporting/requirements#_platforms}
	
	\bibitem{nodejs}
	\emph{node js system requirements}.
	\url{https://www.bing.com/search?q=node+js+system+requirements&qs=n&form=QBRE&sp=-1&lq=0&pq=node+js+system+requirements&sc=11-27&sk=&cvid=66ABA76F06124C6E9D7839509608A22D&ghsh=0&ghacc=0&ghpl=}
	
	\bibitem{ionic}
	\emph{Environment Setup | Ionic Documentation}.
	\url{https://ionicframework.com/docs/intro/environment#:~:text=To%20get%20started%20with%20Ionic%20Framework%2C%20the%20only,a%20free%2C%20batteries-included%20text%20editor%20made%20by%20Microsoft.}
	
	\bibitem{reactjs}
	\emph{How to Install \& Setup React JS on Windows}.
	\url{https://www.knowledgehut.com/blog/web-development/installation-of-react-on-windows}
	
	\bibitem{appinventiv2023}
	Appinventiv.
	\emph{Top 10 frameworks and tools to build Progressive web apps}.
	\emph{Appinventiv}, 2023.
	\url{https://appinventiv.com/blog/top-pwa-development-frameworks/#:~:text=Top%20PWA%20tools%20and%20best%20PWA%20frameworks%201,7%20ScandiPWA%20...%208%20Svelte%20...%20M%C3%A1s%20elementos}
	
\end{thebibliography}

	
\end{document}
